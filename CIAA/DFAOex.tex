\documentclass[dvipdfmx,10pt]{beamer}
\bibliographystyle{junsrt}
\setbeamertemplate{navigation symbols}{}
\setbeamertemplate{footline}[frame number] 
\usepackage{graphicx}
\usepackage{minijs}%和文用
\renewcommand{\kanjifamilydefault}{\gtdefault}%和文用
\usetheme{EastLansing}
\usepackage{tikz}
\usepackage{xspace}
\usepackage{comment}
\usepackage{hyperref}
\usepackage{otf}
\usefonttheme{serif}
\usepackage{ascmac}
%\usepackage[absolute,overlay]{textpos}
\usefonttheme{professionalfonts}
\usepackage{amsmath}
\usepackage{caption}

\usetikzlibrary{lindenmayersystems}
\usetikzlibrary{shapes,arrows}

\usetikzlibrary{positioning,automata}
\usetikzlibrary{decorations.markings}
\usepackage{mathtools}
\usepackage{geometry}
\usetikzlibrary{%
  arrows,%
  positioning,%
  decorations.pathmorphing,%
  decorations.pathreplacing,%
}






\setlength{\columnsep}{3zw}
\makeatletter
\def\presentation#1{\def\@presentation{#1}}
\def\@maketitle{
\begin{flushright}
{\large \@date}% 日付 
\end{flushright}
\begin{center}
{\large \@presentation}% 会名 
\end{center}% 
\begin{center} 
{\LARGE \@title \par}% タイトル 
\end{center}
\begin{center} 
{\large \@author}% 著者 

\end{center}
\par\vskip 1.5em
}
\makeatother

%\beamersetuncovermixins{\opaqueness<1>{25}}{\opaqueness<2->{15}}
\begin{document}
\title{Towards the algorithmic molecular self-assembly of fractals \\by cotranscriptional folding} 
\author{Yusei Masuda, Shinnosuke Seki, and Yuki Ubukata}
\institute{University of Electro-Communications Tokyo}
\date{July 30th, 2018}
\begin{frame}
\maketitle
\end{frame}

%%%%%%%%%%%%%%%%%%%%%%%%%%%%%%%%%%%%%%%%%%%%%%%%%%%%%%%%
\begin{frame}\frametitle{Cotranscriptional folding}
\scalebox{0.85}{

\begin{tikzpicture}
\foreach \x in {12}{
\draw[thick,-latex,red](3,10)-- ++(0:9)-- ++(0:0.5)-- ++(300:1.5)-- ++(180:0.5);
\draw[thick,blue](10,10)-- ++(0:1) node [above] {\LARGE{HA-0}}-- ++(0:1);
\draw[thick,blue](5,10)-- ++(0:1) node [above] {\LARGE{HA-0}}-- ++(0:1);

\uncover<2->{\draw[thick,orange] (\x,10)++(300:1.5)-- ++(180:1.2) node [above, orange] {\LARGE{HA-1b}} -- ++(180:0.8)-- ++(120:1)-- ++(0:2)-- ++(300:1);
\draw[thick,orange,-latex] (\x-2,10)++(300:0.5)-- ++(180:0.5);
}

\uncover<3->{\draw[thick] (\x -2.5,10)++(300:1.5)-- ++(180:2)-- ++(120:1)-- ++(0:2)-- ++(300:1);
\draw[thick,-latex] (\x-4.5,10)++(300:0.5)-- ++(180:0.5);
}

\uncover<4->{\draw[thick,orange] (\x -5,10)++(300:1.5)-- ++(180:1.2) node [above, orange] {\LARGE{HA-0b}} -- ++(180:0.8)-- ++(120:1)-- ++(0:2)-- ++(300:1);
\draw[thick,orange,-latex] (\x-7,10)++(300:0.5)-- ++(180:0.5);
}

\uncover<5->{\draw[thick] (\x -7.5,10) ++(300:1.5)-- ++(120 :1)-- ++(180:1)-- ++(300:2.5)-- ++(0:1)-- ++(120:1.5);
\draw[thick,-latex] (\x-7.5,10)++(300:2)-- ++(0:0.5);
}

\uncover<6->{\draw[thick,orange] (\x -5,10)++(300:3)-- ++(180:1.2) node [above, orange] {\LARGE{HA-0}} -- ++(180:0.8)-- ++(120:1)-- ++(0:2)-- ++(300:1);
\draw[thick,orange,-latex] (\x-5,10)++(300:2)-- ++(0:0.5);
}

\uncover<7->{\draw[thick] (\x -2.5,10)++(300:3)-- ++(180:2)-- ++(120:1)-- ++(0:2)-- ++(300:1);
\draw[thick,-latex] (\x-2.5,10)++(300:2)-- ++(0:0.5);
}

\uncover<8->{\draw[thick,orange] (\x,10)++(300:3)-- ++(180:1.2) node [above , orange] {\LARGE{HA-1}} -- ++(180:0.8)-- ++(120:1)-- ++(0:2)-- ++(300:1);
\draw[thick,orange,-latex] (\x,10)++(300:2)-- ++(0:0.5);
}

\uncover<9->{\draw[thick] (\x +1.5,10) ++(300:3)-- ++(120 :1)-- ++(180:1)-- ++(300:2.5)-- ++(0:1)-- ++(120:1.5);
\draw[thick,-latex] (\x+0.5,10)++(300:4.5)-- ++(180:0.5);
}

\uncover<10->{\draw[thick,orange] (\x,10)++(300:4.5)-- ++(180:1.2) node [above, orange] {\LARGE{HA-0a}} -- ++(180:0.8)-- ++(120:1)-- ++(0:2)-- ++(300:1);
\draw[thick,orange,-latex] (\x-2,10)++(300:4.5)-- ++(180:0.5);
}
\uncover<11->{\draw[thick] (\x -2.5,10)++(300:4.5)-- ++(180:2)-- ++(120:1)-- ++(0:2)-- ++(300:1);
\draw[thick,-latex] (\x-4.5,10)++(300:4.5)-- ++(180:0.5);
}
\uncover<12->{\draw[thick,orange] (\x-5,10)++(300:4.5)-- ++(180:1.2) node [above, orange] {\LARGE{HA-1b}} -- ++(180:0.8)-- ++(120:1)-- ++(0:2)-- ++(300:1);
\draw[thick,orange,-latex] (\x-7,10)++(300:3.5)-- ++(180:0.5);
}
}
\end{tikzpicture}

}

\end{frame}

%%%%%%%%%%%%%%%%%%%%%%%%%%%%%%%%%%%%%%%%%%%%%%%%%%%%%%%%%%%%%%%%%%
\end{document}